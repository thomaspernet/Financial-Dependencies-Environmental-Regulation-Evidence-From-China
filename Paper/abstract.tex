\documentclass[12pt]{article}

\usepackage{amssymb,amsmath,amsfonts,eurosym,geometry,
ulem,graphicx,color,setspace,sectsty,comment,footmisc,
natbib,pdflscape,subfigure,array,hyperref, booktabs,
threeparttable, siunitx, adjustbox}
%https://www.stefaanlippens.net/latex-trick-customizing-captions.html
\usepackage[font=sf, labelfont={sf,bf}, margin=1cm, justification=raggedright,singlelinecheck=false]{caption}

\usepackage[utf8]{inputenc}

%https://www.overleaf.com/learn/latex/Natbib_citation_styles
\usepackage{natbib}
\bibliographystyle{chicago}
\setcitestyle{authoryear,open={(},close={)}}

\graphicspath{ {./images/} }
\normalem



\geometry{left=1.0in,right=1.0in,top=1.0in,bottom=1.0in}

\begin{document}

\begin{titlepage}

\title{Financial Dependencies, Environmental Regulation, and Pollution Intensity: Evidence From China\thanks{We would like to thank the organizers (Etienne Le Rossignol and Candice Yandam) and participants of the Development Economics Sorbonne Informal Research Seminar (DESIR)}}
\author{
Mathilde Maurel\thanks{CNRS, France and Centre d'Economie de la Sorbonne, Université Paris 1 Panthéon-Sorbonne, France} 
\and Thomas Pernet\thanks{Centre d'Economie de la Sorbonne, Université Paris 1 Panthéon-Sorbonne, France,
\href{mailto:t.pernetcoudrier@gmail.com}{email: t.pernetcoudrier@gmail.com} 
%email: \href{t.pernetcoudrier@gmail.com}
}
\and Zhao Ruili\thanks{Shanghai University of International Business and Economics, China}
}

\date{}

\maketitle
\begin{abstract}
\noindent We study how banks' involvement in a firm's financing may be in line with environmental policies pursued by the Chinese central government. Specifically, we evaluate the effectiveness of credit reallocation away from polluting project when the government imposes stringent environmental policies. We combine the industries' financial dependencies with the time and cross-cities variation in policy intensity to identify the causal effect on the sulfur dioxide (SO2) emission. We find that in industries with high reliance on credits, stricter environmental regulations, the SO2 emission are lower. Furthermore, the results suggest that location with strong environmental policies leads firms to seek funding in less restricted areas, confirming the pollution haven hypothesis.
\vspace{0em}\\
\noindent\textbf{Keywords:} Banks, Financial Dependency, Environmental regulation, China\\
\vspace{0em}\\
\noindent\textbf{JEL Codes:} F36, G20, Q53, Q56\\

\bigskip
\end{abstract}
\setcounter{page}{0}
\thispagestyle{empty}
\end{titlepage}
\pagebreak \newpage

\end{document}